\fancyhead{}  % Efface tous les en-têtes existants

\vspace*{1cm}
\begin{center}
    \textbf{\huge{General Introduction}}
\end{center}
\vspace{1cm}

\begin{doublespace}
In today's digital landscape, cybersecurity has become a cornerstone of enterprise operations, particularly in the automotive industry where connected vehicles require robust cryptographic protection. The management of cryptographic resources—certificates and keys that secure communications—presents a critical challenge for organizations seeking to maintain both security and operational continuity.

The Renault Group has recognized the strategic importance of centralizing cryptographic asset management through its Key Management System (KMS) initiative. While secure storage and distribution of cryptographic materials forms the foundation of this system, proactive lifecycle management represents an equally crucial dimension that can prevent service disruptions and security vulnerabilities.

This internship project addresses a fundamental gap in automated cryptographic resource management: the need for intelligent notification systems that can anticipate expiration events and coordinate appropriate organizational responses. The challenge extends beyond simple monitoring to encompass sophisticated stakeholder targeting, multi-channel communication, and seamless enterprise integration.

The work presented details the design and implementation of an automated audit and alert system that transforms reactive resource management into a proactive, intelligence-driven process. Through modern software engineering practices and enterprise-grade technologies, this solution enhances security compliance while reducing operational overhead and risk exposure.

This document chronicles the journey from problem analysis through system design to practical implementation, demonstrating how targeted automation addresses complex organizational challenges in cybersecurity management.
\end{doublespace}